% !TEX root = ../thesis.tex
\chapter{Introduction}
\label{Introduction}
\thispagestyle{empty}

% % If you want, in the beginning of each chapter you can add 
% % a short citation. Some people like to do that. The following lines
% % of code can be used as a template to replicate in each chapter
%\begin{quotation}
%{\footnotesize
%\noindent{\emph{``Citation...''}
%}
%\begin{flushright}
%Author/source
%\end{flushright}
%}
%\end{quotation}
%\vspace{0.5cm}

For the past decade or so, we have been witnessing smartphones taking the world by storm. Since the introduction of smartphones, we have grown used to relying in so many aspects of our lives. We make so many of our decisions based on what the various apps on our smartphones tell us: some apps tell us about good places to eat, others help us book hotels when going abroad and plan our itinerary, and with the improvements in online security, now we can even opt for online banking.\\\\
Naturally, as the number of users of such applications exploded, more and more creators wanted to provide for these users. Therefore, we, as users, have now a plethora of applications to choose from, no matter what the topic is. At first glance, this might seems like a good outcome: more applications mean more options and more customizations. However, that is not really the case: firstly, it is easy for a user to get confused as to which application to use. These applications provide somewhat similar content for a certain topic, and there aren't really too many distinctive features that would allow a user to pick one application over the other. Secondly, these applications sometimes try to provide a tailor-made experience, using some machine learning algorithms, coupled with the previous searches and actions performed by a user. It is worthy to note though that due to the lack of specific contextual data pertaining to the user's conditions, the results risk being rather poor. Due to these reasons, it becomes obvious that an increase in the amount of information available doesn't necessarily cause an increase in knowledge.\\\\
In this project, we try to tackle both of these problems. On one hand, the system we are designing aims to provide a truly \emph{context-aware} platform that takes into consideration as many details as possible concerning the user's needs. On the other hand, we try to incorporate as many data sources as possible, seamlessly integrating their outputs, while keeping in mind that adding additional sources should be simplified as much as possible.\\\\
While these problems might seem trivial at first glance, digging deeper into each of them and trying to implement working solutions reveal the hidden intricacies that make this field such fascinating yet challenging one. We will try now to provide some understanding of the notions of context-awareness and data integrations, as well as where they fit in our project.

\section{What is context?}
\label{sec:context}
One of the most common ways of understanding a term is breaking it down into its elementary components, defining each component separately, and then examine how these components complete one another in the scope of the main term. This is straightforward when looking at the latter part of ``context awareness": being aware of something means to possess knowledge or perception of it. Therefore, context awareness means to be able to know and understand that the ``context" is. However, the big question is what is ``context"? What defines it? What are the factors that make up a ``context", particularly within the domain of mobile applications and their users? The Oxford Dictionaries define the word ``context" as ``the circumstances that form the setting for an event, statement, or idea, and in terms of which it can be fully understood''. Our problems however lies in narrowing down what ``circumstances" are.
\section{What is context awareness?}
\label{sec:contextawareness}
Context awareness means being aware of the different circumstances that define a user's current situation, thus allowing the system to provide results that take such circumstances into consideration. We should also look at a user's context as a dynamic and changing element, as opposed to relatively static measures such as age, gender, or nationality, which define the profile of a user. Some factors include time, location, identity, and current activity in the scope of a user's context. Other factors can be added such as a user's role (the context of a father travelling with his family is different from that of a teacher accompanying a group of students on a field trip). Being able to pinpoint accurately what to include when studying a user's context will go a long way in helping us provide a customized and fitting user experience.

\section{What is Data Integration?}
\label{sec:integration}
If we visit the Android Play Store or the Apple Store and we write a very simple term such as food or restaurant, seemingly endless lists of apps will pop up, and many of these apps will have decent ratings and download counts. If we take a number of these apps and perform the same search on all of them, we will also find long lists of results, with many intersections amongst the various apps. On many occasions, there is not really a clear set of criteria that would allow us to choose one of the results from a list, let alone choosing one of the apps over the others.\\\\
A good scenario for the user would be to have a single app that somehow integrates all of the outputs from the most significant apps, and shows him one comprehensive list of results. An even better scenario would also take the aforementioned context into consideration, and apply it to the list of the results. This would mean that all of the results provided to the user are customized to their specific case; in other words, all of the results represent a pool of valid options for the user, and it would come down to their personal preferences when making a choice.

\section{Putting it together}
\label{sec:together}
In order to have a system that is able to perform these operations, a lot of background work needs to take place. The first component of this system is a backend server able to query multiple services, retrieve results, and integrate them together. It is worthy to note that the responses from different services can possibly have very different structures. Therefore, we need to let the backend system know what kind of response to expect from each service, as well as how to extract the information that we deem relevant.\\\\
The second component is a tool that would allow a system administrator to input the context-specific details of the user. This will be our way of translating the particular needs of a user into a language understood by our system. We need to expose a list of options that is extensive enough to choose from, as the circumstances of users differ greatly, and there are many details that need to be included.\\\\
Finally yet importantly, we need a middleware that is able to take the context representation of the user and apply it to the response received from the backend server. This middleware will have to perform the mapping between the context-related information on one hand, and the collected search results on the other hand. Once the mapping is done, the results are filtered accordingly and presented to the end-user. The innovative part is designing the different components in a generic way, as to allow adapting it to as many different topics and scenarios as possible. Implementing this system will go a long way in providing a truly customized experience to users.\\\\
In what follows, we make use of the example of tourists needing to perform various searches, or asking for recommendations for different activities (where to eat, which events to attend etc...), and they perform these operations through the mobile application installed on their devices by the travel agent. This example includes enough diversity for our purposes one one hand, and on the other hand, there are enough publicly available web services providing relevant information obtainable through relatively simple API calls.

\section{Contribution}
The main contributions we are aiming to achieve by the end of this project are the following:
\begin{itemize}
\item A study of the PerLa framework as a viable data integration solution to be used as a component of CAMUS.
\item A working implementation of the generic web service querying and integration operations.
\item An abstract design, as well as the corresponding implementation of resource storage and query caching.
\end{itemize}
