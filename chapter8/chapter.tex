% !TEX root = ../thesis.tex
\chapter{Conclusion and Future Work}
\label{capitolo8}
\thispagestyle{empty}
\section{Conclusion}
In this work, we have proposed both a high-level design, as well as a low-level, detailed architecture of some of the components of the CAMUS system. The sheer number of details that should be examined and studied in this system, as well as the deep complexity surrounding some of them, renders finalizing all of them within the scope of this thesis nigh impossible. However, every small step gets us closer to having a fully operational framework.\\\\
We have tried to propose an architecture, focusing for the most part on the backend of CAMUS. We have studied the different ways and tools that allow us to perform data integration, and ended up developing our own code, which offered us the biggest amount of freedom when it comes to design and optimization strategies. We have successfully implemented these designs, and the early tests we have performed seemed to be quite promising.\\\\
Our data integration model, along with the implementation, were able to fetch data from various heterogeneous sources, integrate the different outputs, making sure that we are extracting everything we need. Adding new services proved to be simple as well: if we follow the guidelines when adding a new service, and provide to the parsers the information pertaining to these new services (as descriptors), then the system will seamlessly integrate the new endpoints into the relevant wrapper, without any extra coding required whatsoever.\\\\
The storage and caching strategies were able to reduce greatly the running time required to perform a full cycle between the backend and the web services. This cycle includes building the requests that need to be sent, customizing them to match what the web services expect to receive in these requests, retrieve and parse the responses, and finally integrate them into one homogenous structure. These operations are quite costly in terms of running time, which is a big constraint when we are considering mobile applications. We were able however to meet this constraint by adding query caching functionalities, as well as saving the end result of the data integration operations in our own database structure.\\\\
Finally yet importantly, we were able to adhere to the context-independence paradigm when designing the various components: the data integration model, as well as the storage mechanisms, work no matter what kind of data we are dealing with, as long as suitable descriptors are provided. We want CAMUS to be a framework that can work for as wide of a variety of domains as possible. To that, the code should not depend on the kind of resources we are dealing with. Naturally, however, in some cases, a domain specialist might be needed to provide insight on how which attributes to include in the CDT and the resource schema. The technical implementation however does not depend on such factors.
\section{Future Work}
Having presented a solid foundation for the back-end component, work can now be done to integrate this component with the other elements. Once the work on the individual components is completed, the system as a whole can be tested to see the accuracy of our context model, and our data integration module can be also further improved. Concerning the backend, the resource schema structure is as of now hard-coded into the back-end component. However, it would be better if the resource schema is defined in an external descriptor, using a similar logic to what we did to describe the web services. Doing this will increase the modularity of the back-end, and it will facilitate editing existing resources and adding new ones.\\\\
Moreover, the current implementation lacks the user interfaces that would expose the various functionalities that are available. In particular, an interface that exposes that resource schema and the context dimension tree to the administrator -allowing him to make changes as necessary- would remove the need to write manually the necessary code instead. This also applies to the interface that will be used by the middle agent to project the different factors that make up the end-user' context onto the context dimension tree; a suitable interface which shows the entirety of the context dimension tree without overwhelming the middle agent, and which allows easily editing and pruning needs to be developed.\\\\
The final interface in need of work is the actual application that will be used by the end-user. This interface represents the culmination of all of the work that is taking place behind the scene. It is also worthy to note that this interface is the \emph{only} interface seen by the end-user, and therefore, it has to be intuitive, simple, and it has to adhere to the latest UI guidelines.\\\\
Additionally, in order to increase the usability and robustness of the system, a formal methodology should be proposed, and this methodology would clearly define the steps required to have the system up and running as intended. Doing this for every agent in the system help make sure that everyone knows their role, and what kind of input is expected from them, as well as how to supply this input. Once we have all of this finalized, we can move on to testing real-life scenarios, and only then would we able to judge the accuracy of our model, and the efficiency of our components.
