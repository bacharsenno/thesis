% !TEX root = ../thesis.tex
\chapter{Code}
\label{appendiceA}
\thispagestyle{empty}
The following code takes a JSON schema of a response received from web services, and generates the corresponding XML descriptor. The XML descriptor can then be interpreted by PerLa, and then used to communicate with this web service. The JSON schema is generated using one of the many publicly available schema generators, such as the one hosted on \hyperref[https://github.com/gonvaled/jskemator]{https://github.com/gonvaled/jskemator}\\
\begin{lstlisting}[language=Python]

import simplejson
from pprint import pprint
from collections import defaultdict, OrderedDict
from operator import itemgetter
import sys
import getopt

Elementary = {"integer", "float", "string", ""}
Basic = {"integer", "float", "string"}
value_object = defaultdict(list)
cleaned_value_object = OrderedDict()
parent_list = dict()
depth_array = dict()
parent = "root"
maxDepth = 0

def process_options(argv):
	filename = None
	try:
		opts, args = getopt.getopt(argv[1:], "f:s", ['filename='])
	except getopt.GetoptError, err:
	# print help information and exit:
		print str(err)
		sys.exit(2)
	for o, a in opts:
		if o in ("-f", "--filename"):
			filename = a
	return filename
	
def generate_value(key, value):
	if key == "formattedAddress":
		print ''
	if value["type"] == "array":
		if value["properties"] and value["properties"][0]["type"] in Basic:
			return '<js:value name="' + key + '" type="' + value["properties"][0]["type"] + '" qualifier="list" />'
		return '<js:value name="' + key + '" type="' + key + '" qualifier="list" />'
	elif value["type"] == "object":
		return '<js:value name="' + key + '" type="' + key + '" />'
	else:
		return '<js:value name="' + key + '" type="' + value["type"] + '" />'
		
def isElementary(o):
	if o["type"] in Elementary:
		return True
	elif o["type"] == "object":
		return False
	elif o["type"] == "array":
		i = o["properties"]
		if not i:
			return True
		return isElementary(i[0])
		
def isBaseObject(o):
	for key, value in o.iteritems():
		if value == None:
			value = dict()
			value["type"] = "string"
		if isElementary(value) or key in Elementary:
			continue
		else:
			return False, key
	return True, True

def getObject(o):
	if o["type"] == "array":
		return o["properties"][0]["properties"]
	else:
		return o["properties"]

def parseObject(o, parent, depth):
	depth = depth + 1
	global maxDepth
	if depth > maxDepth:
		maxDepth = depth
	if parent in parent_list:
		parent_list[parent] = parent_list[parent] + 1
	else:
		parent_list[parent] = 1
	parent = parent + ":::" + str(parent_list[parent])
	for key, value in o.iteritems():
		if key == "groups":
			print ''
		if key == "id":
			continue
		if value == None:
			value = dict()
			value["type"] = "string"
			o[key] = value
		if isElementary(value) or key in Elementary:
			value_object[parent].append(generate_value(key, value))
			continue
		val, k = isBaseObject(getObject(value))
		if val == True:
			parseObject(getObject(value), key, depth)
			Elementary.add(key)
			value_object[parent].append(generate_value(key, value))
		else:
			temp = getObject(getObject(value)[k])
			parseObject(temp, k, depth + 1)
			Elementary.add(k)
			parseObject(getObject(value), key, depth)
			value_object[parent].append(generate_value(key, value))
	depth_array[parent] = depth


def cleanupValues():
	for key, value in value_object.iteritems():
		if ":::" in key:
			k = key.split(":::")[0]
			if k in cleaned_value_object.keys():
				for val in value:
					if val not in cleaned_value_object[k]:
						cleaned_value_object[k].append(val)
			else:
				cleaned_value_object[k] = value

def reorderValues():
	global cleaned_value_object
	for key, value in cleaned_value_object.iteritems():
		for key2, value2 in depth_array.iteritems():
			k = key2.split(":::")[0]
			if key == k:
				if type(cleaned_value_object[key][-1]) == str:
					cleaned_value_object[key].append(value2)
				elif cleaned_value_object[key][1] < value2:
					cleaned_value_object[key][1] = value2
	cleaned_value_object = sorted(cleaned_value_object.items(), key=lambda x: x[1][-1], reverse=True)

def printValues(valueArray):
	for value in valueArray:
		print "\t" + value

def generateXML():
	for key, value in cleaned_value_object:
		if key == None:
			print ''
		if key == "root":
			continue
		else:
			print '<js:object id="' + key + '">'
			printValues(value[:-1])
			print '</js:object>'
	print '<js:object id="root">'
	printValues(value_object.get("root:::1")[:-1])
	print '</js:object>'

def main():
	filename = process_options(sys.argv)
	with open(filename, 'r') as f:
		data = simplejson.load(f)
	data = data["properties"]
	parseObject(data, "root", 0)
	cleanupValues()
	reorderValues()
	generateXML()
			
if __name__ == '__main__':
	main()
\end{lstlisting}